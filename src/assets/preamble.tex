% Copyright (C) 2014-2020 by Thomas Auzinger <thomas@auzinger.name>
% changed 2022 by Sascha Hunold

\documentclass[draft,final]{vutinfth} % Remove option 'final' to obtain debug information.

% Load packages to allow in- and output of non-ASCII characters.
\usepackage{fontspec}
\setmonofont[Scale=MatchLowercase]{DejaVu Sans Mono}
% \usepackage{lmodern}        % Use an extension of the original Computer Modern font to minimize the use of bitmapped letters.
\usepackage[T1]{fontenc}    % Determines font encoding of the output. Font packages have to be included before this line.
\usepackage[utf8]{inputenc} % Determines encoding of the input. All input files have to use UTF8 encoding.

% Extended LaTeX functionality is enables by including packages with \usepackage{...}.
\usepackage{amsmath}    % Extended typesetting of mathematical expression.
\usepackage{amssymb}    % Provides a multitude of mathematical symbols.
\usepackage{mathtools}  % Further extensions of mathematical typesetting.
\usepackage{microtype}  % Small-scale typographic enhancements.
\usepackage[inline]{enumitem} % User control over the layout of lists (itemize, enumerate, description).
\usepackage{multirow}   % Allows table elements to span several rows.
\usepackage{booktabs}   % Improves the typesettings of tables.
\usepackage{subcaption} % Allows the use of subfigures and enables their referencing.
\usepackage[usenames,dvipsnames,table]{xcolor} % Allows the definition and use of colors. This package has to be included before tikz.
\usepackage{nag}       % Issues warnings when best practices in writing LaTeX documents are violated.
\usepackage{todonotes} % Provides tooltip-like todo notes.
\setuptodonotes{inline}

% \usepackage[acronym,toc]{glossaries} % Enables the generation of glossaries and lists fo acronyms. This package has to be included last.
% hunsa
\usepackage[binary-units]{siunitx}
\usepackage{xspace}
\usepackage{my_macros}

% hunsa optional for algorithms (change this if you want to use other algorithms packages)
% \usepackage{algorithmicx}
%\usepackage[ruled,linesnumbered,algochapter]{algorithm2e} % Enables the writing of pseudo code.

\usepackage{algorithm}
\usepackage{algorithmicx}
\usepackage{algpseudocode}

\usepackage{listings}
\usepackage{xcolor}

\definecolor{light-blue}{HTML}{6b85dd}
\definecolor{dark-blue}{HTML}{4266d5}
\definecolor{light-red}{HTML}{d66661}
\definecolor{dark-red}{HTML}{c93d39}
\definecolor{light-green}{HTML}{6bab5b}
\definecolor{dark-green}{HTML}{3b972e}
\definecolor{light-purple}{HTML}{aa7dc0}
\definecolor{dark-purple}{HTML}{945bb0}
\definecolor{codeblock-background}{gray}{0.96}
\definecolor{codeblock-border}{gray}{0.8}
%

% images etc.
\usepackage{graphicx}
\usepackage{svg}
\usepackage{pdfpages}
\usepackage[export]{adjustbox}
%

\usepackage{hyperref}  % Enables cross linking in the electronic document version. This package has to be included second to last.
\hypersetup{
    pdfpagelabels,
    bookmarks,
    hyperindex,
    unicode = true,
}

% Some internal link targets are implemented with \label, some with \hypertarget,
% but they require different links. This inserts a \hyperref if a corresponding label exists,
% and \hyperlink if it doesn't.
\makeatletter
\def\hyperlinkref#1#2{\@ifundefined{r@#1}{\hyperlink{#1}{#2}}{\hyperref[#1]{#2}}}
\makeatother
%

% listings
\usepackage{minted}
\setminted{
    breaklines = true,
    fontsize = \small,
    frame = none,
    bgcolor = codeblock-background,
    rulecolor=codeblock-border,
}
%

\lstset{
  frame=single,
  basicstyle=\footnotesize\ttfamily,
  showstringspaces=false
}


% Define convenience functions to use the author name and the thesis title in the PDF document properties.
\newcommand{\authorname}{Valentin Bogad} % The author name without titles.
\newcommand{\thesistitle}{Guidelines For Performance Optimization In Julia} % The title of the thesis. The English version should be used, if it exists.

% Set PDF document properties
\hypersetup{
    pdfpagelayout   = TwoPageRight,           % How the document is shown in PDF viewers (optional).
    linkbordercolor = {Melon},                % The color of the borders of boxes around crosslinks (optional).
    pdfauthor       = {\authorname},          % The author's name in the document properties (optional).
    pdftitle        = {\thesistitle},         % The document's title in the document properties (optional).
    % pdfsubject      = {Subject},              % The document's subject in the document properties (optional).
    pdfkeywords     = {julia, optimization} % The document's keywords in the document properties (optional).
}

\setpnumwidth{2.5em}        % Avoid overfull hboxes in the table of contents (see memoir manual).
\setsecnumdepth{subsection} % Enumerate subsections.

\nonzeroparskip             % Create space between paragraphs (optional).
\setlength{\parindent}{0pt} % Remove paragraph identation (optional).


% \makeindex      % Use an optional index.
% \makeglossaries % Use an optional glossary.
%\glstocfalse   % Remove the glossaries from the table of contents.

% Set persons with 4 arguments:
%  {title before name}{name}{title after name}{gender}
%  where both titles are optional (i.e. can be given as empty brackets {}).
\setauthor{}{\authorname}{}{male}
\setadvisor{Associate Prof. Dipl.-Inform. Dr.rer.nat.}{Sascha Hunold}{}{male}

% For bachelor and master theses:
% \setfirstassistant{Pretitle}{Forename Surname}{Posttitle}{male}
% \setsecondassistant{Pretitle}{Forename Surname}{Posttitle}{male}
% \setthirdassistant{Pretitle}{Forename Surname}{Posttitle}{male}

% For dissertations:
% \setfirstreviewer{Pretitle}{Forename Surname}{Posttitle}{male}
% \setsecondreviewer{Pretitle}{Forename Surname}{Posttitle}{male}

% For dissertations at the PhD School and optionally for dissertations:
% \setsecondadvisor{Pretitle}{Forename Surname}{Posttitle}{male} % Comment to remove.

% Required data.
\setregnumber{01633662}
\setdate{09}{11}{2023} % Set date with 3 arguments: {day}{month}{year}.
\settitle{\thesistitle}{Guidelines For Performance Optimization In Julia} % Sets English and German version of the title (both can be English or German). If your title contains commas, enclose it with additional curvy brackets (i.e., {{your title}}) or define it as a macro as done with \thesistitle.
%\setsubtitle{Optional Subtitle of the Thesis}{Optionaler Untertitel der Arbeit} % Sets English and German version of the subtitle (both can be English or German).

% Select the thesis type: bachelor / master / doctor / phd-school.
% Bachelor:
\setthesis{bachelor}
%
% Master:
%\setthesis{master}
%\setmasterdegree{dipl.} % dipl. / rer.nat. / rer.soc.oec. / master
%
% Doctor:
%\setthesis{doctor}
%\setdoctordegree{rer.soc.oec.}% rer.nat. / techn. / rer.soc.oec.
%
% Doctor at the PhD School
%\setthesis{phd-school} % Deactivate non-English title pages (see below)

% For bachelor and master:
%\setcurriculum{Media Informatics and Visual Computing}{Medieninformatik und Visual Computing} % Sets the English and German name of the curriculum.
\setcurriculum{Software \& Information Engineering}{Software \& Information Engineering} % Sets the English and German name of the curriculum.


% For dissertations at the PhD School:
%\setfirstreviewerdata{Affiliation, Country}
%\setsecondreviewerdata{Affiliation, Country}

% https://aty.sdsu.edu/bibliog/latex/floats.html
% Alter some LaTeX defaults for better treatment of figures:
    % See p.105 of "TeX Unbound" for suggested values.
    % See pp. 199-200 of Lamport's "LaTeX" book for details.
    %   General parameters, for ALL pages:
    \renewcommand{\topfraction}{0.9}	% max fraction of floats at top
    \renewcommand{\bottomfraction}{0.8}	% max fraction of floats at bottom
    %   Parameters for TEXT pages (not float pages):
    \setcounter{topnumber}{2}
    \setcounter{bottomnumber}{2}
    \setcounter{totalnumber}{4}     % 2 may work better
    \setcounter{dbltopnumber}{2}    % for 2-column pages
    \renewcommand{\dbltopfraction}{0.9}	% fit big float above 2-col. text
    \renewcommand{\textfraction}{0.07}	% allow minimal text w. figs
    %   Parameters for FLOAT pages (not text pages):
    \renewcommand{\floatpagefraction}{0.7}	% require fuller float pages
	% N.B.: floatpagefraction MUST be less than topfraction !!
    \renewcommand{\dblfloatpagefraction}{0.7}	% require fuller float pages

	% remember to use [htp] or [htpb] for placement

\begin{document}

\frontmatter % Switches to roman numbering.
% The structure of the thesis has to conform to the guidelines at
%  https://informatics.tuwien.ac.at/study-services

\addtitlepage{naustrian} % German title page (not for dissertations at the PhD School).
\addtitlepage{english} % English title page.
\addstatementpage

\begin{acknowledgements*}
I want to thank my friends and family for their repeated questions of progress, without which I would have lost motivation long ago,
even though I was often complaining about those same questions. This thesis truly would not have been written were it not for the support given by them,
and I am incredibly grateful for that.
Additionally, my thanks goes out to the Julia Community, who have shown me through example that dynamism and performance are two goals that
need not be in contradiction.
Finally, I also want to thank my advisor Sascha Hunold for the incredible patience and support given during the long period it has taken me to finish this thesis.
\end{acknowledgements*}

\begin{kurzfassung*}
Julia ist eine relative junge Sprache, welche Allgemeines und Wissenschaftliches Rechnen mit hoher Performance bedient, gleichzeitig
aber auch einen dynamischen und interaktiven Ansatz um Code zu entwickeln im Fokus hat. Diese Flexibilität erlaubt es EntwicklerInnen,
schnell neuen Code zu entwickeln und diesen effizient auch auf gro{\ss}en Rechenclustern zu verwenden. Optimierter Julia Code erreicht
State Of The Art Geschwindigkeit in einer Vielzahl von Domänen, unter anderem Machine Learning, Automatic Differentiation,
Wirtschaftswissenschaften, während zudem die Möglichkeit gegeben wird, lokal explorative Analysen durchzuführen. Diese Thesis behandelt
eine Vielzahl von Strategien für die Optimierung von Julia Code, von Typbasierten Problematiken über Algorithmische Optimierungen hin zu
Analyse von Assembly Instruktionen in hei{\ss}en Schleifen.
\end{kurzfassung*}

\begin{abstract*}
Julia is a relatively young language, targeting general \& scientific computing with high performance, all while focusing on a dynamic and
interactive approach to developing code. This flexibility allows for fast iteration during development, while at the same time scaling to
large clusters. As a result, optimized Julia code achieves state of the art performance in a number of domains, including machine
learning, automatic differentiation, economic modeling and others, while sometimes giving the opportunity for developers to move from a
cluster/job based workflow to a fully interactive workflow allowing exploratory analysis. In this thesis, we aim to provide a number of general strategies
for optimizing all kinds of Julia code, reaching from high level type based difficulties, over algorithmic optimizations down to
assembly level analysis of hot loops.
\end{abstract*}

% Select the language of the thesis, e.g., english or naustrian.
\selectlanguage{english}

% Add a table of contents (toc).
\tableofcontents* % Starred version, i.e., \tableofcontents*, removes the self-entry.
\cleardoublepage % Start list of tables on the next empty right hand page.

% Use an optional list of figures.
%\listoffigures % Starred version, i.e., \listoffigures*, removes the toc entry.
% \cleardoublepage % Start list of tables on the next empty right hand page.

% Use an optional list of tables.
% \listoftables % Starred version, i.e., \listoftables*, removes the toc entry.
% \cleardoublepage % Start list of tables on the next empty right hand page.

% Use an optional list of algorithms.
% \listofalgorithms
% \addcontentsline{toc}{chapter}{List of Algorithms}

% Switch to arabic numbering and start the enumeration of chapters in the table of content.
\mainmatter

